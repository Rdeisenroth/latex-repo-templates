\documentclass[
    inlineshortcut=java, % Befehl \inlinejava{<code>} Konfigurieren
    corporatedesign, % TU-Design
    boxarc, % Abgerundete Ecken bei den Boxen
    fop, % FOP-Vorlage benutzen
]{algoexercise}

%%------------%%
%%--Packages--%%
%%------------%%

% \usepackage{audutils}
% \usepackage{fopbot}

%%---------------------------%%
%%--Dokumenteneinstellungen--%%
%%---------------------------%%

\duedate{2025-XX-XX}
% \author{<Author1>\and <Author2>\and <Author3>} % Übungsblattbetreuer
\subtitle{Prof. Schlau Meier}
\dozent{Prof. Schlau Meier} % chktex 12
\fachbereich{Informatik}
\semester{Wintersemester 2025/26} % z.B. SoSe 2022 oder WiSe 2022/2023
\sheetnumber{12} % Einstellige Nummern werden mit 0 aufgefüllt
\slides{08} % Die Relevanten Foliensätze
\topics{Streams} % Für das Übungsblatt relevante Themengebiete
\title[<Kreativer Titel>]{Übungsblatt \getSheetnumber{}}
\version{1.0-SNAPSHOT}
\graphicspath{{./pictures/}}

%%----------------------------%%
%%--Stilistische Anpassungen--%%
%%----------------------------%%

\ConfigureHeadline{
    headline={algo-min}
}

%%-------------------------%%
%%--Beginn des Dokumentes--%%
%%-------------------------%%

\begin{document}%

    %%-----------%%
    %%--Titelei--%%
    %%-----------%%

    \maketitle{}

    %%-------------%%
    %%--H-Übungen--%%
    %%-------------%%

    \hue{Hausübung \getSheetnumber{}}{\getShorttitle{}}{\getPointsTotal{}}

    %\tableofcontents

    %%------------------------------%%
    %%--Verbindliche Anforderungen--%%
    %%------------------------------%%

    \UseSnippet{exercise-introduction-fop-2425}

    %%--------------%%
    %%--Einleitung--%%
    %%--------------%%

    <Einleitungstext>

    \clearpage{}

    %%------------------------%%
    %%--Beginn der Hausübung--%%
    %%------------------------%%

    % Aufgabe 1
    \begin{task}[points=auto]{<Aufgabentitel>}\label{ex:H1}
        \begin{subtask*}[points=1]{<Teilaufgabe>}\label{ex:H1.1}
            <Aufgabenstellung>
        \end{subtask*}
        \begin{subtask*}[points=2]{<Teilaufgabe>}\label{ex:H1.2}
            <Aufgabenstellung>
        \end{subtask*}
    \end{task}
\end{document}
